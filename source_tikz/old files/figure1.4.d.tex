\documentclass[border = 0.2cm]{standalone}
 
% Required package
\usepackage{tikz}
 
\begin {document}
 
% Layer A neurons'number
\newcommand{\numlayerA}{12} 
 
% Layer B neurons'number
\newcommand{\numlayerB}{12}  

% Layer C neurons'number
\newcommand{\numlayerC}{20}   
 
% Layer C neurons'number
\newcommand{\numlayerD}{10} 
 
\begin{tikzpicture}
 
% Layer A
\foreach \i in {1,...,\numlayerA}
{
    \node[circle, draw, minimum size = 5mm] (layerA\i) at (0,-\i*2) {};
}

% Layer B
\foreach \i in {1,...,\numlayerB}
{
    \node[circle, draw, minimum size = 5mm ] (layerB\i) at (5,-\i+7*-1) {};
}

% Layer C
\foreach \i in {1,...,\numlayerC}
{
    \node[circle, draw, minimum size = 5mm ] (layerC\i) at (10,-\i+3*-1) {};
}

% Layer D
\foreach \i in {1,...,\numlayerD}
{
    \node[circle, draw, minimum size = 5mm ] (layerD\i) at (15,-\i+7*-1) {};
}

% Connect neurons between hidden layers - Can use is if we can have error handling in tixz
%\foreach \i [evaluate=\i as \in using int(\i+1) ] in {1,2,3}
%{
%    \foreach \j  in {1,...,\numlayerC}
%    {
%        \foreach \k  in {1,...,\numlayerC}
%        {
%            \draw[] (layer\i\j) -- (layer\in\k);
%        }
%    }
%} 

%connect Layer A and Layer B
\foreach \i in {1,...,\numlayerA}
{
    \foreach \j in {1,...,\numlayerB}
    {
      \draw[] (layerA\i) -- (layerB\j);   
    }
} 

%connect Layer B and Layer C
\foreach \i in {1,...,\numlayerB}
{
    \foreach \j in {1,...,\numlayerC}
    {
      \draw[] (layerB\i) -- (layerC\j);   
    }
} 

%connect Layer C and Layer D
\foreach \i in {1,...,\numlayerC}
{
    \foreach \j in {1,...,\numlayerD}
    {
      \draw[] (layerC\i) -- (layerD\j);   
    }
} 

\end{tikzpicture}
 
\end {document}