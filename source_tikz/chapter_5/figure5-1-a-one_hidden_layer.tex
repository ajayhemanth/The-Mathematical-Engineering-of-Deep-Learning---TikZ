\documentclass[tikz, border=50pt]{standalone}

\usepackage{tikz}
\usepackage{medl_colors}
\usetikzlibrary{shapes.multipart, shapes.geometric, arrows.meta}
\usetikzlibrary{matrix, calc, positioning,fit}

\begin {document}
 
% Layer A neurons'number
\newcommand{\numlayerA}{4} 
 
% Layer B neurons'number
\newcommand{\numlayerB}{5}  

% Layer C neurons'number
\newcommand{\numlayerC}{1}   
  
\begin{tikzpicture}
 
% Layer A
\foreach \i in {1,...,\numlayerA}
{
    \node[greenshape, circle, minimum size = 1cm] (layerA\i) at (0,-\i*2-1) {\Large $x_\i$};
}

% Layer B
\foreach \i in {1,...,\numlayerB}
{
    \node[blueshape, circle, minimum size = 1cm] (layerB\i) at (4,-\i*2) {};
}

% Layer C
\foreach \i in {1,...,\numlayerC}
{
    \node[redshape, circle, minimum size = 1cm] (layerC\i) at (8,-6) {\Large $\hat y$};
}

%connect Layer A and Layer B
\foreach \i in {1,...,\numlayerA}
{
    \foreach \j in {1,...,\numlayerB}
    {
      \draw[-Triangle, thickline] (layerA\i) -- (layerB\j);   
    }
} 

%connect Layer B and Layer C
\foreach \i in {1,...,\numlayerB}
{
    \foreach \j in {1,...,\numlayerC}
    {
      \draw[-Triangle, thickline] (layerB\i) -- (layerC\j);   
    }
} 

\node[above of=layerA1, align=center, node distance=2.5cm] {\Large Input\\\Large Layer};
\node[above of=layerB1, align=center, node distance=1.5cm] {\Large Hidden\\\Large Layer};
\node[above of=layerC1, align=center, node distance=5.5cm] {\Large Output\\\Large Layer};

\end{tikzpicture}
 
\end {document}