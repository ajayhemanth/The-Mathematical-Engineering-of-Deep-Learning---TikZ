\documentclass[border=1cm]{standalone}
\usepackage{tikz}
\usepackage{chemfig}
\usepackage{medl_colors}

\begin{document}
\setchemfig{atom sep=2em,bond style={line width=1pt, black!20}}   
% \chemfig{
% *6(
%     C(=O)-
%     {\color{border-blue}N}(-C(-[4]{\color{fill-grey}H})(-[6]{\color{fill-grey}H})(-[8]{\color{fill-grey}H}))-
%     C(*5(-{\color{border-blue}N}=C-{\color{border-blue}N}(-C(-[8]{\color{fill-grey}H})(-[1]{\color{fill-grey}H})(-[2]{\color{fill-grey}H}))-))=
%     C-
%     C(=O)-
%     {\color{border-blue}N}(-C(-[5]{\color{fill-grey}H})(-[4]{\color{fill-grey}H})(-[3]{\color{fill-grey}H}))-
% )}

\begin{tikzpicture}
\node at (0,0){\chemfig{
*6(
    C(=O)-
    {\color{border-blue}N}(-C(-[4]{\color{fill-grey}H})(-[6]{\color{fill-grey}H})(-[8]{\color{fill-grey}H}))-
    C(*5(-{\color{border-blue}N}=C-{\color{border-blue}N}(-C(-[8]{\color{fill-grey}H})(-[1]{\color{fill-grey}H})(-[2]{\color{fill-grey}H}))))=
    C-
    C(=O)-
    {\color{border-blue}N}(-C(-[5]{\color{fill-grey}H})(-[4]{\color{fill-grey}H})(-[3]{\color{fill-grey}H}))
)}};
\draw [line width = 1pt, black!20](-0.76,0.25) -- (-0.76,-0.07);
\draw [line width = 1pt, black!20](0.65,0.5) -- (0.95,0.6);
\end{tikzpicture}
\end{document}
